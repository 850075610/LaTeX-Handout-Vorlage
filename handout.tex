\documentclass{handout}
\SetHandoutTitle{Cultural turn}
\SetUniversitaet{Europa Universität Viadrina}
\SetDatum{\today}
\SetFakultaet{Fakultät der Kulturwissenschaften}
\SetSemester{Sommersemester 2013}
\SetDozent{Professor Moriarty}
\SetReferent{Max Mustermann}
\SetMatrikelnr{0815}
\SetSeminar{Einführung in die Kulturgeschichte (Tutorium)}
\SetLiteratur{handout}

\hypersetup{
    pdftitle={Handout \HandoutTitle},
    pdfauthor={\Referent},
    pdfsubject={},
    pdfkeywords={},
    pdfcreator={LaTeX, hyperref, KOMA-Script}, % Ersteller
  }
\begin{document}
\maketitle

\begin{longtable}[ht]{p{0.20\textwidth} p{0.74\textwidth}}
\multicolumn{2}{p{1\textwidth}}{Meine \underline{persönliche Definition} des Begriffes Cultural turn ist, 
\begin{quote}
	\enquote{\textit{Die (Ver-)Änderunge vom WAS und/oder WIE innerhalb der Kulturwissenschaftlichen Betrachtungen.}}
\end{quote}}
\\
\textbf{Definition (h.M.)} & Abkehr von einer Kultur der Elite hinzu einer Populärkultur.\\
\textbf{Beginn} & \textit{strittig} irgendwo zwischen dem frühen 20. Jahrhundert und 1940/50\\
\textbf{Höhepunkt} & bisher ca. 1960 mit Gründung der Kulturwissenschaften \\
\textbf{Ende} & meiner Ansicht nach gibt es keine Ende, da der Begriff des Cultural turn ein Oberbegriff für alle turns ist und sich ständig neue turns bilden b.z.w. entwickeln.\\
\textbf{Ursprung} & \textit{strittig} vom Linguistic turn oder von den sich geänderten Fragestellungen innerhalb der Sozialwissenschaften hergeleitet.\\
\textbf{Turns} & 
\begin{noindlist}
	\item Linguistic turn
	\item Postcolonial turn
	\item Iconic turn
	\item Performative turn
	\item Spatial turn
	\item Emotional turn
	\item Reflexive turn
	\item Translational turn
	\item Interpretive turn
	\item (Gender turn)
\end{noindlist}\\
\end{longtable}
\newpage
\makeliteratur
\end{document}